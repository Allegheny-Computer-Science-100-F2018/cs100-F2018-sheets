\documentclass[11pt]{article}

% NOTE: The "Edit" sections are changed for each assignment

% Edit these commands for each assignment

\newcommand{\assignmentduedate}{October 19}
\newcommand{\assignmentassignedate}{October 10}
\newcommand{\assignmentnumber}{One}

\newcommand{\labyear}{2018}
\newcommand{\assignedday}{Wednesday}
\newcommand{\dueday}{Friday}
\newcommand{\labtime}{9:00 am}

\newcommand{\assigneddate}{Announced: \assignedday, \assignmentassignedate, \labyear{} at \labtime{}}
\newcommand{\duedate}{Quiz: \dueday, \assignmentduedate, \labyear{} at \labtime{}}

% Edit these commands to give the name to the main program

\newcommand{\mainprogram}{\lstinline{DisplayOutput}}
\newcommand{\mainprogramsource}{\lstinline{src/main/java/labone/DisplayOutput.java}}

% Edit this commands to describe key deliverables

\newcommand{\reflection}{\lstinline{writing/reflection.md}}

% Commands to describe key development tasks

% --> Running gatorgrader.sh
\newcommand{\gatorgraderstart}{\command{gradle grade}}
\newcommand{\gatorgradercheck}{\command{gradle grade}}

% --> Compiling and running program with gradle
\newcommand{\gradlebuild}{\command{gradle build}}
\newcommand{\gradlerun}{\command{gradle run}}

% Commands to describe key git tasks

% NOTE: Could be improved, problems due to nesting

\newcommand{\gitcommitfile}[1]{\command{git commit #1}}
\newcommand{\gitaddfile}[1]{\command{git add #1}}

\newcommand{\gitadd}{\command{git add}}
\newcommand{\gitcommit}{\command{git commit}}
\newcommand{\gitpush}{\command{git push}}
\newcommand{\gitpull}{\command{git pull}}

\newcommand{\gitcommitmainprogram}{\command{git commit src/main/java/labone/DisplayOutput.java -m "Your
descriptive commit message"}}

% Use this when displaying a new command

\newcommand{\command}[1]{``\lstinline{#1}''}
\newcommand{\program}[1]{\lstinline{#1}}
\newcommand{\url}[1]{\lstinline{#1}}
\newcommand{\channel}[1]{\lstinline{#1}}
\newcommand{\option}[1]{``{#1}''}
\newcommand{\step}[1]{``{#1}''}

\usepackage{pifont}
\newcommand{\checkmark}{\ding{51}}
\newcommand{\naughtmark}{\ding{55}}

\usepackage{listings}
\lstset{
  basicstyle=\small\ttfamily,
  columns=flexible,
  breaklines=true
}

\usepackage{fancyhdr}

\usepackage[margin=1in]{geometry}
\usepackage{fancyhdr}

\pagestyle{fancy}

\fancyhf{}
\rhead{Computer Science 100}
\lhead{Quiz \assignmentnumber{}}
\rfoot{Page \thepage}
\lfoot{\duedate}

\usepackage{titlesec}
\titlespacing\section{0pt}{6pt plus 4pt minus 2pt}{4pt plus 2pt minus 2pt}

\newcommand{\guidetitle}[1]
{
  \begin{center}
    \begin{center}
      \bf
      CMPSC 100\\Computational Expression\\
      Fall 2018\\
      \medskip
    \end{center}
    \bf
    #1
  \end{center}
}

\begin{document}

\thispagestyle{empty}

\guidetitle{Quiz \assignmentnumber{} Study Guide \\ \assigneddate{} \\ \duedate{}}

\section*{Introduction}

\noindent
The quiz will be ``closed notes'' and ``closed book'' and it will cover the
following materials. Please review the ``Course Schedule'' on the Web site for
the course to see the content and slides that we have covered to this date.
Students may post questions about this material to our Slack team.

\begin{itemize}

  \itemsep 0in

  \item Chapter One in Lewis \& Loftus (i.e., ``Introduction to Computation and
    Programming'')

  \item Chapter Two in Lewis \& Loftus, Sections 2.1--2.6 (i.e., ``Data and
    Expressions'')

  \item Chapter Three in Lewis \& Loftus, Sections 3.1--3.11 (i.e., ``Using
    Classes and Objects'')

  \item Using the basic commands in the Linux operating system; editing in {\tt
    gvim}, compiling and executing programs in Linux; knowledge of the basic
    commands for using {\tt git} and GitHub.

  \item Your class notes and lecture slides, labs 1--5, practicals 1--3, and the
    handouts from lab.

\end{itemize}

\noindent The quiz will be a mix of questions that have a form such as fill in
the blank, short answer, true/false, and completion. The emphasis will be on the
following topics:

\vspace*{-.05in}
\begin{itemize}

  \itemsep 0in

  \item Fundamental concepts in computing and the Java language (e.g.,
    definitions and background)

  \item Practical laboratory techniques (e.g., editing, compiling, and running
    programs; effectively using files and directories; correctly using GitHub
    through the command-line {\tt git} program)

  \item Understanding Java programs (e.g., given a short, perhaps even one line,
    source code segment written in Java, understand what it does and be able to
    precisely describe its output).

  \item Composing Java statements and programs, given a description of what
    should be done. Students should be completely comfortable writing short
    source code statements that are in nearly-correct form as Java code. While
    your program may contain small syntactic errors, it is not acceptable to
    ``make up'' features of the Java programming language that do not exist in
    the language itself---so, please do not call a ``{\tt
    solveQuestionThree()}'' method!

\end{itemize}

\noindent No partial credit will be given for questions that are true/false,
completion, or fill in the blank. Minimal partial credit may be awarded for the
questions that require a student to write a short answer. You are strongly
encouraged to write short, precise, and correct responses to all of the
questions. When you are taking the quiz, you should do so as a ``point
maximizer'' who first responds to the questions that you are most likely to
answer correctly for full points. Please keep the time limitation in mind as you
are absolutely required to submit the examination at the end of the class period
unless you have written permission for extra time from a member of the Learning
Commons. Students who do not submit their quiz on time will have their overall
point total reduced. Please see the course instructor if you have questions
about any of these policies.

% \section*{Reminder Concerning the Honor Code}

Students are required to fully adhere to the Honor Code during the completion of
this quiz. More details about the Allegheny College Honor Code are provided on
the syllabus available from the course web site. Students are strongly
encouraged to carefully review the full statement of the Honor Code before
taking this quiz. So as to maintain your own integrity and that of the College,
the Honor Code requires you complete your own work during the quiz. If you do
not understand Allegheny College's Honor Code, please schedule an office hours
meeting with the course instructor.

% \begin{quote}
%   The Academic Honor Program that governs the entire academic program at
%   Allegheny College is described in the Allegheny Academic Bulletin. The Honor
%   Program applies to all work that is submitted for academic credit or to meet
%   non-credit requirements for graduation at Allegheny College. This includes all
%   work assigned for this class (e.g., examinations, laboratory assignments, and
%   the final project). All students who have enrolled in the College will work
%   under the Honor Program.  Each student who has matriculated at the College has
%   acknowledged the following pledge:
% \end{quote}

% \begin{quote}
%   I hereby recognize and pledge to fulfill my responsibilities, as defined in
%   the Honor Code, and to maintain the integrity of both myself and the College
%   community as a whole.
% \end{quote}

% \noindent It is understood that an important part of the learning process in
% any course, and particularly one in computer science, derives from thoughtful
% discussions with teachers and fellow students. Such dialogue is encouraged.
% However, it is necessary to distinguish carefully between the student who
% discusses the principles underlying a problem with others and the student who
% produces assignments that are identical to, or merely variations on, someone
% else's work.  While it is acceptable for students in this class to discuss
% their programs, data sets, and reports with their classmates, deliverables
% that are nearly identical to the work of others will be taken as evidence of
% violating the \mbox{Honor Code}.

\section*{Detailed Review of Content}

The listing of topics in the following subsections is not exhaustive; rather,
it serves to illustrate the types of concepts that students should study as
they prepare for the quiz. Please see the course instructor during office hours
if you have questions about any of the content listed in this section.

\vspace*{-.1in}
\subsection*{Chapter One}

\begin{itemize}

  \itemsep 0in

  \item Basic understanding of computer hardware and software
  \item Computer number systems (e.g., binary and decimal)
  \item Purpose for and steps of the fetch-decode-execute cycle in the CPU
  \item Layout of and access techniques for computer memory
  \item Knowledge of computer networking methods and programs
  \item Basic syntax and semantics of the Java programming language
  \item Examples of valid and invalid Java variable names
  \item Appropriate ways to add comments to a Java program
  \item Input(s) and output(s) of the Java compiler and virtual machine
  \item How to use \command{gradle} to build and run a Java program

\end{itemize}

\vspace*{-.2in}
\subsection*{Chapter Two}

\begin{itemize}

  \itemsep 0in

  \item Using escape sequences to control the output of Java programs
  \item Ways to perform input and output in a Java program
  \item The variety of data types available to Java programmers
  \item The trade-offs associated with using different data types
  \item The declaration of and assignment of values to variables
  \item Operators and operator precedence in Java expressions
  \item Techniques for converting variables from one data type to another

\end{itemize}

\vspace*{-.2in}
\subsection*{Chapter Three}

\begin{itemize}

  \itemsep 0in

  \item The steps for creating a new instance of a Java class
  \item How to use technical diagrams to visualize an object in memory
  \item The meaning of the term ``alias'' in a Java program
  \item The creation and use of Strings in the Java programming language
  \item The ways in which Java packages promote high-quality programming
  \item The variety of ways in which you can create and use random numbers
  \item How to call and use the methods provided by the {\tt java.lang.Math} class
  \item Ways in which programs create formatted output in a terminal window
  \item Computer graphics and related topics such as pixels and screen resolution
  \item The coordinate system used to display computer graphics in Java
  \item The use of the RGB system for specifying colors in Java programs
  \item How to call Java methods to display colored shapes on the screen
  \item How to use arithmetic expressions to calculate colors and layouts

\end{itemize}

\end{document}
