\documentclass[11pt]{article}

% NOTE: The "Edit" sections are changed for each assignment

% Edit these commands for each assignment

\newcommand{\assignmentduedate}{December 14}
\newcommand{\assignmentassignedate}{November 27}
\newcommand{\assignmentnumber}{Eleven}

\newcommand{\labyear}{2018}
\newcommand{\labday}{Tuesday}
\newcommand{\labdueday}{Friday}
\newcommand{\labtime}{2:30 pm}
\newcommand{\labduetime}{11:00 pm}

\newcommand{\assigneddate}{Assigned: \labday, \assignmentassignedate, \labyear{} at \labtime{}}
\newcommand{\duedate}{Due: \labdueday, \assignmentduedate, \labyear{} by \labduetime{}}

% Edit these commands to give the name to the main program

\newcommand{\mainprogram}{\lstinline{MandelbrotBlackAndWhite}}
\newcommand{\mainprogramsource}{\lstinline{src/main/java/labten/MandelbrotBlackAndWhite.java}}

\newcommand{\secondprogram}{\lstinline{MandelbrotColor}}
\newcommand{\secondprogramsource}{\lstinline{src/main/java/labten/MandelbrotColor.java}}

% Edit this commands to describe key deliverables

\newcommand{\reflection}{\lstinline{writing/reflection.md}}
\newcommand{\timefile}{\lstinline{writing/time.md}}
\newcommand{\colorfile}{\lstinline{writing/color.md}}
\newcommand{\readme}{\lstinline{README.md}}

% Commands to describe key development tasks

% --> Running gatorgrader.sh
\newcommand{\gatorgraderstart}{\command{gradle grade}}
\newcommand{\gatorgradercheck}{\command{gradle grade}}

% --> Compiling and running program with gradle
\newcommand{\gradlebuild}{\command{gradle build}}
\newcommand{\gradlerun}{\command{gradle run}}

% Commands to describe key git tasks

\newcommand{\gitcommitfile}[1]{\command{git commit #1}}
\newcommand{\gitaddfile}[1]{\command{git add #1}}

\newcommand{\gitadd}{\command{git add}}
\newcommand{\gitcommit}{\command{git commit}}
\newcommand{\gitpush}{\command{git push}}
\newcommand{\gitpull}{\command{git pull}}

\newcommand{\gitaddmainprogram}{\command{git add src/main/java/labten/MandelbrotColorManager.java}}
\newcommand{\gitcommitmainprogram}{\command{git commit src/main/java/labten/MandelbrotColorManager.java -m "Your
descriptive commit message"}}

% Use this when displaying a new command

\newcommand{\command}[1]{``\lstinline{#1}''}
\newcommand{\program}[1]{\lstinline{#1}}
\newcommand{\url}[1]{\lstinline{#1}}
\newcommand{\channel}[1]{\lstinline{#1}}
\newcommand{\option}[1]{``{#1}''}
\newcommand{\step}[1]{``{#1}''}

\usepackage{pifont}
\newcommand{\checkmark}{\ding{51}}
\newcommand{\naughtmark}{\ding{55}}

\usepackage{listings}
\lstset{
  basicstyle=\small\ttfamily,
  columns=flexible,
  breaklines=true
}

\usepackage{fancyhdr}
\usepackage{fancyvrb}

\usepackage[margin=1in]{geometry}
\usepackage{fancyhdr}

\pagestyle{fancy}

\fancyhf{}
\rhead{Computer Science 100}
\lhead{Final Project}
\rfoot{Page \thepage}
\lfoot{\duedate}

\usepackage{titlesec}
\titlespacing\section{0pt}{6pt plus 4pt minus 2pt}{4pt plus 2pt minus 2pt}

\newcommand{\labtitle}[1]
{
  \begin{center}
    \begin{center}
      \bf
      CMPSC 100\\Computational Expression\\
      Fall 2018\\
      \medskip
    \end{center}
    \bf
    #1
  \end{center}
}

\begin{document}

\thispagestyle{empty}

\labtitle{Final Project \\ \assigneddate{} \\ \duedate{}}

\section*{Introduction}

Throughout the semester, you have explored the fundamentals of computer science
and Java programming by studying, in a hands-on fashion, topics such as data and
expressions, the use and creation of Java classes, conditionals and loops, and
arrays and lists. This final project invites you to explore, in greater detail,
a real-world application of computer science. You will learn more about how to
use, implement, test, and evaluate different types of real-world computer
software. Since you will complete the final project with a partner, you will
also learn more about how GitHub and the Git version control system can
effectively support collaborative software development.

Your project should result in a detailed report and and at least three Java
source files, in addition to written materials and technical diagrams that
highlight the key contributions of your work. This technical report should
include a description of why the chosen topic is important and discuss the
implementation and/or experimentation that you undertook. The written material
should be precise, formal, appropriately formatted, grammatically correct,
informative, and interesting. The source code that you write must be carefully
documented and tested. If you install and use existing computer software (e.g.,
a Java library for creating computer graphics), the steps for installation and
use should be clearly documented in your report. Also, the report must explain
the steps to run your own Java program. Finally, your report must detail the
work completed by each member of your partnership; individual contributions
should also be reflected in the Git repository's log. In addition to writing the
aforementioned final report in Markdown, you will also use Markdown to write and
submit a project proposal and a status update at the stated intermediate
deadlines.

\section*{Suggestions for Success}

\begin{itemize}
  \setlength{\itemsep}{0pt}

\item {\bf Use the laboratory computers}. The computers in this laboratory feature specialized software for completing
  this course's laboratory and practical assignments. If it is necessary for you to work on a different machine, be sure
  to regularly transfer your work to a laboratory machine so that you can check its correctness. If you cannot use a
  laboratory computer and you need help with the configuration of your own laptop, then please carefully explain its
  setup to a teaching assistant or the course instructor when you are asking questions.

\item {\bf Follow each step carefully}. Slowly read each sentence in the assignment sheet, making sure that you
  precisely follow each instruction. Take notes about each step that you attempt, recording your questions and ideas and
  the challenges that you faced. If you are stuck, then please tell a teaching assistant or instructor what assignment
  step you recently completed.

\item {\bf Regularly ask and answer questions}. Please log into Slack at the start of a laboratory or practical session
  and then join the appropriate channel. If you have a question about one of the steps in an assignment, then you can
  post it to the designated channel. Or, you can ask a student sitting next to you or talk with a teaching assistant or
  the course instructor.

\item {\bf Store your files in GitHub}. As in the past laboratory assignments, you will be responsible for storing all
  of your files (e.g., Java source code and Markdown-based writing) in a Git repository using GitHub Classroom. Please
  verify that you have saved your source code in your Git repository by using \command{git status} to ensure that
  everything is updated. You can see if your assignment submission meets the established correctness requirements by
  using the provided checking tools on your local computer and by checking the commits in GitHub.

\item {\bf Keep all of your files}. Don't delete your programs, output files, and written reports after you submit them
  through GitHub; you will need them again when you study for the quiz and the final exam and work on the other
  practical assignments and final project checkpoints.

\item {\bf Back up your files regularly}. All of your files are regularly backed-up to the servers in the Department of
  Computer Science and, if you commit your files regularly, stored on GitHub. However, you may want to use a flash
  drive, Google Drive, or your favorite backup method to keep an extra copy of your files on reserve. In the event of
  any type of system failure, you are responsible for ensuring that you have access to a recent backup copy of all your
  files.

\item {\bf Explore teamwork and technologies}. While certain aspects of the laboratory assignments will be challenging
  for you, each part is designed to give you the opportunity to learn both fundamental concepts in the field of computer
  science and explore advanced technologies that are commonly employed at a wide variety of companies. To explore and
  develop new ideas, you should regularly communicate with your team and/or the teaching assistants and tutors.

\item {\bf Hone your technical writing skills}. Computer science assignments require to you write technical
  documentation and descriptions of your experiences when completing each task. Take extra care to ensure that your
  writing is interesting and both grammatically and technically correct, remembering that computer scientists must
  effectively communicate and collaborate with their team members and the tutors, teaching assistants, and course
  instructor.

\item {\bf Review all of your past laboratory and practical assignments}. Now that you have completed many prior
  assignments, please review all of your prior work to ensure that you understand the concepts needed to explore
  real-world applications of computer science.

\item {\bf Review the Honor Code on the syllabus}. While you may discuss your assignments with others, copying source
  code or writing is a violation of Allegheny College's Honor Code.

\end{itemize}

\section*{Reading Assignment}

To ensure that you are best prepared to complete this final project, please
review all of the chapters that we have covered up to the release date of this
assignment. As we cover new material during the remainder of the semester (e.g.,
arrays and exception handling), you are also encouraged to review that content
as it may better enable you to complete a high-quality final project.

\section*{Accessing the Final Project Assignment on GitHub}

To access the laboratory assignment, you should go into the
\channel{\#announcements} channel in our Slack team and find the announcement
that provides a link for it. Now, your team should look into the
\channel{\#announcements} channel on Slack and claim the next available number
for your team. Next, the team leader (i.e., the person you elect to complete
this step) will create their team with the name
\command{Computer-Science-100-Fall-2018-Lab-10-Group-<group label>} and then
accept the laboratory assignment and see that GitHub Classroom created a new
GitHub repository for your team to access the assignment's starting materials
and to store the completed version of your assignment. Note that the team leader
will have to type their group and lab details into a text field. For instance,
if the team leader was in the second group then that person would type ``Group 2
for Lab 10'' into the text field. At this point, each additional member of the
team can accept the assignment through GitHub. Please make sure that each of
your team members joins the correct team---you must be in the team with your
number. Unless you provide the instructor with documentation of the extenuating
circumstances that you are facing, not accepting the assignment means that you
automatically receive a failing grade for it. Please see the course instructor
with any questions.

Before you move to the next step of this assignment, please make sure that you
read all of the content on the web site for your new GitHub repository, paying
close attention to the technical details about the commands that you will type
and any extra steps that you must take. Now you are ready to download the
starting materials to your laboratory computer. Click the ``Clone or download''
button and, after ensuring that you have selected ``Clone with SSH'', please
copy this command to your clipboard. To enter the correct directory you should
now type \command{cd cs100F2018}. Next, you can type the command \command{ls}
and see that there are some files or directories inside of this directory. By
typing \command{git clone} in your terminal and then pasting in the string that
you copied from the GitHub site you will download all of the code for this
assignment. For the previous example, a student would run a \command{git clone}
command in the terminal in this fashion:

\begin{lstlisting}
  git clone git@github.com:Allegheny-Computer-Science-100-F2018/computer-science-100-fall-2018-lab-10-group-2-for-lab-10.git
\end{lstlisting}

After this command finishes, you can use \command{cd} to change into the new
directory. If you want to \step{go back} one directory from your current
location, then you can type the command \command{cd ..}. You may continue to use
the \command{cd} and \command{ls} commands to explore the files that you
automatically downloaded from GitHub. Please note that the course instructor
expects students to implement and evaluate all of the Java source code needed to
complete their proposed project. As such, there are no provided Java source code
files for this assignment. With that said, you will need to edit the four
Markdown files in the \program{writing/} directory by the stated deadline. Here
are some additional tips to help you design, implement, and test the Java source
code for the final project:

\begin{itemize}
  \item Make sure that your Java source code contains declarations that place it in the correct package.
  \item Given the name that you picked for your package, ensure that your directories are correct.
  \item Update the \program{build.gradle} file so that it contains the correct package and Java class name.
  \item Update the \program{settings.gradle} file so that it also references the correct directory.
  \item Make sure that your \program{build.gradle} file is setup to support graphical and/or textual output.
  \item Remember that all of your Java source code must meet Google's programming standards.
  \item Recall that there are no GatorGrader code checks since each sinal project is different.
  \item In GatorGrader's absence, your team should establish correctness checks for your source code.
\end{itemize}

\section*{Description of the Topics}

Each partnership is invited to pick one of the following projects.  Please note that a partnership selecting the
student-designed project must first discuss the idea with the course instructor, during today's laboratory session, and
receive feedback and then final approval. Please note that you and your partner are fully responsible for ensuring the
feasibility of the project that you propose.

\begin{enumerate}

  \item {\bf Cryptography and Cryptanalysis}: Explore a topic in the fields that make up the ``art and science of
    sending and decoding secret messages''. This project invites you and your partner to implement and test several
    cryptography and/or cryptanalysis systems. To start, you should investigate, implement, and test ciphers such as
    the Caesar and Vigenere ciphers. Then, you should use your ciphers to demonstrate that you can successfully send
    secret messages through, for instance, an email server. In addition to creating and testing these Java programs,
    your report should include a detailed explanation of how your chosen algorithms work.

  \item {\bf Computer Graphics}: Potentially using your textbook's sections on computer graphics as a starting point,
    implement a complete program that displays graphics and colors. Students who select this project should consider
    ways in which the graphics can, for instance, represent realistic entities, support interactivity, and properly
    adhere to artistic principles of color, light, and layout. In addition to furnishing the project's source code, your
    report should include a detailed artistic, technical, and/or mathematical commentary on the final graphics.

  \item {\bf Interactive Storytelling}: Leveraging your knowledge of file and console input and output, create a
    computer program that allows a person to engage with a ``choose your own adventure''-style story or game. After
    reading in commands from the user, your program will display information about the world subject to exploration and
    act on the user's request. Along with creating and testing your program, your report should fully describe your
    interactive story and how you used, for instance, conditional logic and iteration to create it.

  \item {\bf Computer Simulation}: There are many physical or biological phenomena that are challenging to study in
    either a laboratory or a natural setting. As such, natural and social scientists may implement computer simulations
    that model these phenomena, often leading to additional insights into how the real systems operate. This project
    invites you and your partner to pick a phenomenon that you would like to simulate and then implement a full-featured
    simulation in the Java programming language. Your chosen phenomenon for simulation might involve the throwing of a
    ball, the vibration of a string, chemical reactions, or the interaction between predator and prey animals. In
    addition to implementing and running your simulation, your report should completely describe the phenomenon that
    your program models and explain how your implementation uses, for example, methods for random number generation.

  \item {\bf Performance Evaluation}: Since it is often important to implement computer software that exhibits
    acceptable time and space overheads, this project invites students to use and/or extend performance evaluation
    software like \program{System.nanoTime()}. After finding, reading, and understanding textbooks and research papers
    on this topic, students who pick this project should identify a focus area and create a benchmarking tool. The final
    version of the framework should allow students to measure the performance of computations in Java, report those
    measurements to the user of the benchmarks, and support informed decision making about design and implementation
    trade-offs. Along with including the source code of the benchmarking framework, this project invites students to
    write a performance evaluation report.

  \item {\bf Software Testing}: In response to the fact that real-world software often contains serious defects, this
    project encourages students to learn more about techniques that can find program errors before they are delivered to
    end-users. Students who choose this topic will investigate software testing tools, such as JUnit, and then create
    a software system that includes a test suite. In addition to implementing Java classes and methods that provide the
    main functionality, you will, as part of this project, also write tests that assess the correctness of the
    aforementioned methods. If your team picks this project, you should submit programs and test suites in
    addition to a report that explains your approach to software testing.

  \item {\bf Student-Designed Project}: Students will develop an idea for their own project that focuses on one or more
    real-world topics in the field of computer science. After receiving the course instructor's approval for your idea,
    you will complete the project and report on your results.

\end{enumerate}

\section*{Project Requirements}

To ensure that you and your partner have mastered the concepts discussed in this course, your project's source code
should adhere to the following requirements. These requirements may be modified, at the discretion of the course
instructor, only if a student receives prior permission and documents this approval in the final report and the source
code. Without prior approval, all submitted programs should contain at least three Java classes that consist in total of
at least five methods in addition to the {\tt main} method. The final project's source code should include examples of
the declaration and use of at least three different data types (e.g., {\tt int}, {\tt String}, and {\tt boolean}). The
program should also include conditional logic, in the form of {\tt if} or {\tt switch} statements, and one or more
iteration constructs, expressed as {\tt while}, {\tt do-while}, and {\tt for} loops. Finally, the programs must also use
either an array, an {\tt ArrayList}, or both of these ``containers'' for variables.

\section*{Summary of the Required Deliverables}

\noindent Students do not need to submit printed source code or technical writing for any assignment in this course.
Instead, this assignment invites you to submit, using GitHub, the following deliverables.

\begin{enumerate}

  \setlength{\itemsep}{0in}

  \item Completed, fully commented, and properly formatted versions of all Java
    source code files.

  \item A four-paragraph written proposal, saved in the file
    \program{writing/proposal.md}, with an informative title, an abstract, a
    description of the main idea, an initial listing of the tasks that you must
    complete, and a plan that you and your partner will follow to complete the
    work.

  \item A one-paragraph status update, saved in the file
    \program{writing/update.md}, that explains what your team has already
    implemented and the steps that you will take to finish your program.

  \item A detailed final project report, saved in the file
    \program{writing/report.md}, that documents, in a project-specific fashion,
    how you designed, implemented, and evaluated your system.

  \item Stored in \reflection{}, a three-paragraph reflection on the commands
    that you typed in \command{gvim} or \command{atom} and the terminal window.
    This Markdown-based document should explain the input, output, and behavior
    of each command and the challenges that you confronted when using it. For
    every challenge that you encountered, please explain your solution for it.
    This document should also explain how your team collaborated to finish the
    assignment, with each team member writing their own paragraph inside of this
    Markdown file.

  \item A commit log in your GitHub repository that clearly shows that the team
    members effectively collaborated. That is, the commit log should show that
    commits were evenly made by all team members; the photograph of each member
    should appear in the commit log.

\end{enumerate}

\noindent
You must complete all of the aforementioned deliverables by the following due dates:

\begin{enumerate}

  \item {\bf Project Assigned and Project Proposal:} Tuesday, November 27, 2018

    After meeting with the course instructor and your partner, pick a topic for
    your final project. Remember, if your team selects the student-designed
    project, you must first have your project approved by the course instructor.
    Next, make sure that you create a GitHub repository that can be accessed by
    the instructor. Finally, write and submit a four-paragraph proposal for your
    project. While you can use the project descriptions on the previous pages as
    a starting point, your proposal should have an informative title, an
    abstract, a description of the main idea, an initial listing of the tasks to
    complete, and a plan for your completion of the work.

  \item {\bf Status Update and Project Demonstration}: Tuesday, December 4, 2018
    by 2:30 pm

    As you continue working on your project, please submit a two-paragraph
    status update in printed form and through your Git repository. In addition,
    you should give a demonstration, during the laboratory session, highlighting
    the most important code that you have finished.

  \item {\bf Final Project Due Date}: Friday, December 14, 2018 by 11:00 pm

    You should submit the final version of your project through your team's
    GitHub repository. This submission should include all of the relevant source
    code and output, the written reports, and any additional materials that will
    demonstrate the success of your project. While you are encouraged to turn in
    the final project before the examination session starts on the due date,
    students must submit the completed assignment no later than 11:00 pm on the
    due date.

\end{enumerate}

\section*{Evaluation of Your Final Project Assignment}

Using a report that the instructor shares with you through the commit log in GitHub, you will privately received a grade
on this assignment and feedback on your submitted deliverables. Your grade for the assignment will be a function of the
whether or not it was submitted in a timely fashion and if your repository contains correct versions of all of the
required Markdown and source code files. Other factors will also influence your final grade on the assignment. The
instructor will also evaluate the accuracy of both your technical writing and the comments in your source code. Please
see the instructor if you have questions about the evaluation of this final project assignment.

All team members will receive the same baseline grade for the final project. If there are extenuating circumstances in
which one or more of the team members do not effectively collaborate to complete this assignment, then the course
instructor will adjust the grade of specific team members so that it is higher or lower than the baseline grade, as is
fair and necessary. Please see the instructor if you do not understand how he assigns grades for collaborative
assignments. Finally, in adherence to the Honor Code, students should only complete this assignment with their team
members. Deliverables (e.g., Java source code or Markdown-based technical writing) that are nearly identical to the work
of outsiders of your team will be taken as evidence of violating the \mbox{Honor Code}. In summary, please see the
course instructor with any questions about this final project policy.

As previously noted, the instructor will also evaluate the effectiveness with which your team completed this assignment.
The instructor will assess your team's effectiveness as {\bf excellent} when:

\vspace*{-.5em}

\begin{enumerate}
  \setlength{\itemsep}{0pt}

  \item All team members contribute correct, useful Java source code and/or Markdown documentation that supports the
    creation of a working and meaningful Java computer program.

  \item All team members participate in the laboratory discussions, sharing status updates and ideas.

  \item All team members attend the classroom discussions about your current status on the project.

  \item All team members regularly communicate with others in a respectful and informative manner.

  \item All team members work with others and the instructor to resolve any differences that arise.

\end{enumerate}

\vspace*{-.5em}

The instructor will assess team effectiveness as {\bf good} when the team members fulfill the aforementioned standards
most of the time. The team's effectiveness will be assessed as {\bf average} when some, but not the majority, of the
time the team members make meaningful contributions to the project, according to these set standards. The course
instructor will assess team effectiveness as {\bf below average} when, as a whole, the team does not function according
to the above standards.

\end{document}
