\documentclass[11pt]{article}

% NOTE: The "Edit" sections are changed for each assignment

% Edit these commands for each assignment

\newcommand{\assignmentduedate}{October 23}
\newcommand{\assignmentassignedate}{October 20}
\newcommand{\assignmentnumber}{Six}

\newcommand{\labyear}{2017}
\newcommand{\labdueday}{Monday}
\newcommand{\labassignday}{Friday}
\newcommand{\labtime}{9:00 am}

\newcommand{\assigneddate}{Assigned: \labassignday, \assignmentassignedate, \labyear{} at \labtime{}}
\newcommand{\duedate}{Due: \labdueday, \assignmentduedate, \labyear{} at \labtime{}}

% Edit these commands to give the name to the main program

\newcommand{\mainprogram}{\lstinline{YearCheckerMain}}
\newcommand{\mainprogramsource}{\lstinline{src/main/java/practicalsix/YearCheckerMain.java}}

\newcommand{\secondprogram}{\lstinline{YearChecker}}
\newcommand{\secondprogramsource}{\lstinline{src/main/java/practicalsix/YearChecker.java}}

% Edit this commands to describe key deliverables

\newcommand{\reflection}{\lstinline{writing/reflection.md}}

% Commands to describe key development tasks

% --> Running gatorgrader.sh
\newcommand{\gatorgraderstart}{\command{./gatorgrader.sh --start}}
\newcommand{\gatorgradercheck}{\command{./gatorgrader.sh --check}}

% --> Compiling and running program with gradle
\newcommand{\gradlebuild}{\command{gradle build}}
\newcommand{\gradlerun}{\command{gradle run}}

% Commands to describe key git tasks

% NOTE: Could be improved, problems due to nesting

\newcommand{\gitcommitfile}[1]{\command{git commit #1}}
\newcommand{\gitaddfile}[1]{\command{git add #1}}

\newcommand{\gitadd}{\command{git add}}
\newcommand{\gitcommit}{\command{git commit}}
\newcommand{\gitpush}{\command{git push}}
\newcommand{\gitpull}{\command{git pull}}

\newcommand{\gitcommitmainprogram}{\command{git commit src/main/java/practicalsix/YearCheckerMain.java -m "Your
descriptive commit message"}}

% Use this when displaying a new command

\newcommand{\command}[1]{``\lstinline{#1}''}
\newcommand{\program}[1]{\lstinline{#1}}
\newcommand{\url}[1]{\lstinline{#1}}
\newcommand{\channel}[1]{\lstinline{#1}}
\newcommand{\option}[1]{``{#1}''}
\newcommand{\step}[1]{``{#1}''}

\usepackage{pifont}
\newcommand{\checkmark}{\ding{51}}
\newcommand{\naughtmark}{\ding{55}}

\usepackage{listings}
\lstset{
  basicstyle=\small\ttfamily,
  columns=flexible,
  breaklines=true
}

\usepackage{fancyvrb}
\usepackage{color}

\usepackage{fancyhdr}

\usepackage[margin=1in]{geometry}
\usepackage{fancyhdr}

\pagestyle{fancy}

\fancyhf{}
\rhead{Computer Science 111}
\lhead{Practical Assignment \assignmentnumber{}}
\rfoot{Page \thepage}
\lfoot{\duedate}

\usepackage{titlesec}
\titlespacing\section{0pt}{6pt plus 4pt minus 2pt}{4pt plus 2pt minus 2pt}

\newcommand{\labtitle}[1]
{
  \begin{center}
    \begin{center}
      \bf
      CMPSC 111\\Introduction to Computer Science I\\
      Fall 2017\\
      \medskip
    \end{center}
    \bf
    #1
  \end{center}
}

\begin{document}

\thispagestyle{empty}

\labtitle{Practical \assignmentnumber{} \\ \assigneddate{} \\ \duedate{}}

\section*{Objectives}

To continue practicing the use of GitHub to access the files for a practical assignment. Additionally, to practice using
the Ubuntu operating system and software development programs such as a ``terminal window'' and the ``GVim text
editor''. You will continue to practice using Slack to support communication with the teaching assistants and the course
instructor. Next, you will create and then call methods that determine if certain events occurred during the year input
from the file. For this task you will use {\tt if/else} statements and boolean logic operators such as ``{\tt \&\&}'' or
``{\tt ||}''. Finally, you will continue to learn more about creating and using Java classes and objects.

\section*{Suggestions for Success}

\begin{itemize}
  \setlength{\itemsep}{0pt}

\item {\bf Use the laboratory computers}. The computers in this laboratory feature specialized software for completing
  this course's Laboratory and practical assignments. If it is necessary for you to work on a different machine, be sure
  to regularly transfer your work to a laboratory machine so that you can check its correctness. If you cannot use a
  laboratory computer and you need help with the configuration of your own laptop, then please carefully explain its
  setup to a teaching assistant or the course instructor when you are asking questions.

\item {\bf Follow each step carefully}. Slowly read each sentence in the assignment sheet, making sure that you
  precisely follow each instruction. Take notes about each step that you attempt, recording your questions and ideas and
  the challenges that you faced. If you are stuck, then please tell a teaching assistant or instructor what assignment
  step you recently completed.

\item {\bf Regularly ask and answer questions}. Please log into Slack at the start of a laboratory or practical session
  and then join the appropriate channel. If you have a question about one of the steps in an assignment, then you can
  post it to the designated channel. Or, you can ask a student sitting next to you or talk with a teaching assistant or
  the course instructor.

\item {\bf Store your files in GitHub}. As in all of your past assignments, you will be responsible for storing
  all of your files (e.g., Java source code and Markdown-based writing) in a Git repository using GitHub Classroom.
  Please verify that you have saved your source code in your Git repository by using \command{git status} to ensure that
  everything is updated. You can see if your assignment submission meets the established correctness requirements by
  using the provided checking tools on your local computer and in checking the commits in GitHub.

\item {\bf Keep all of your files}. Don't delete your programs, output files, and written reports after you submit them
  through GitHub; you will need them again when you study for the quizzes and examinations and work on the other
  laboratory, practical, and final project assignments.

\item {\bf Back up your files regularly}. All of your files are regularly backed-up to the servers in the Department of
  Computer Science and, if you commit your files regularly, stored on GitHub. However, you may want to use a flash
  drive, Google Drive, or your favorite backup method to keep an extra copy of your files on reserve. In the event of
  any type of system failure, you are responsible for ensuring that you have access to a recent backup copy of all your
  files.

\item {\bf Explore teamwork and technologies}. While certain aspects of these assignments will be challenging for you,
  each part is designed to give you the opportunity to learn both fundamental concepts in the field of computer science
  and explore advanced technologies that are commonly employed at a wide variety of companies. To explore and develop
  new ideas, you should regularly communicate with your team and/or the teaching assistants and tutors.

\item {\bf Hone your technical writing skills}. Computer science assignments require to you write technical
  documentation and descriptions of your experiences when completing each task. Take extra care to ensure that your
  writing is interesting and both grammatically and technically correct, remembering that computer scientists must
  effectively communicate and collaborate with their team members and the tutors, teaching assistants, and course
  instructor.

\item {\bf Review the Honor Code on the syllabus}. While you may discuss your assignments with others, copying source
  code or writing is a violation of Allegheny College's Honor Code.

\end{itemize}

\section*{Reading Assignment}

If you have not done so already, please read all of the relevant ``GitHub Guides'', available at
\url{https://guides.github.com/}, that explain how to use many of the GitHub's features. In particular, please make sure
that you have read guides such as ``Mastering Markdown'' and ``Documenting Your Projects on GitHub''; each of them will
help you to understand how to use both GitHub and GitHub Classroom. Focusing on the content about creating and using
Java objects and writing conditional logic, you should review Chapters 1 through 4 and Sections 5.1 and 5.3 in the
textbook.

\section*{Using Conditional Logic in a Java Program}

To access the practical assignment, you should go into the \channel{\#announcements} channel in our Slack team and find
the announcement that provides a link for it. Copy this link and paste it into your web browser. Now, you should accept
the practical assignment and see that GitHub Classroom created a new GitHub repository for you to access the
assignment's starting materials and to store the completed version of your assignment. Specifically, to access your new
GitHub repository for this assignment, please click the green ``Accept'' button and then click the link that is prefaced
with the label ``Your assignment has been created here''. If you accepted the assignment and correctly followed these
steps, you should have created a GitHub repository with a name like
``Allegheny-Computer-Science-111-Fall-2017/computer-science-111-fall-2017-practical-6-gkapfham''. Unless you provide the
instructor with documentation of the extenuating circumstances that you are facing, not accepting the assignment means
that you automatically receive a failing grade for it. Please follow the steps from the previous laboratory assignments
for finding your ``home base'' for this practical assignment; see the instructor if you are stuck on getting started.

This practical assignment is inspired by a quotation is from Allegheny College's Alma Mater. In one of the classes for
this assignment, {\tt YearChecker.java}, you will write methods that, for the file's input from {\tt
YearCheckerMain.java}, determine which of the following events occurs that year: \begin{itemize}

\item
it is designated as a leap year

\item
the emergence of the 17-year cicadas (more specifically, Brood II)

\item
it is predicted to be a peak year of sunspot activity

\end{itemize}

\noindent A year is a {\em leap year\/} if it is divisible by 4, {\em unless\/} it is a century year. If it is a century
year, it is a leap year if it is divisible by 400. For instance, 1968 and 1972 are leap years since they are divisible
by 4; 1967 and 1970 are not. The year 2000 is a leap year because it is divisible by 400; however, 1900 is not (even
though 1900 is divisible by 4---century years are treated differently).

\noindent The 17-year cicadas emerge from underground every 17 years. There are several ``broods''---the one in which we
have interest emerged most recently in 2013. That is, any year that differs from 2013 by a multiple of 17 is also an
emergence year for Brood II (e.g., 2040, 1996, 1928, and 3713).

\noindent Sunspot activity usually peaks every 11 years. The year 2013 was supposed to be such a ``solar max'' year.
This means that any year that differs from 2013 by a multiple of 11 should also be a solar max year. For instance, 2002,
2024, and 1793 are all years predicted to exhibit peak activity.

Figure~\ref{mad} contains the output from running a program like the one you must implement. You should study the
comments in the \mainprogramsource{} to see each step that you have to implement. You should also look at the
\secondprogramsource{} to see the methods that will ultimately contain the conditional logic. After finishing the both
of these files, you should repeatedly test you program to make sure that it is creating the correct textual output. This
will involve you editing the input file and then building and running the program and checking the output to ensure that
it produces different values and that the checks are correct. Don't forget that this practical assignment requires you
to understand and edit two different files called \mainprogramsource{} and \secondprogramsource{}. This means that you
must have correct formatting and documentation in both of these files; check the {\tt README.md} file for a statement of
other checks. You should also be able to draw a diagram explaining the relationship between the two source code files.

\begin{figure}[tb]
\begin{Verbatim}[commandchars=\\\{\}]
Gregory M. Kapfhammer Thu Oct 19 23:39:16 EDT 2017

I will read in a year between 1000 and 3000.
Okay, I read in the year 1452.

Is the year 1452 a leap year? Yes
Is the year 1452 a cicada emergence year? Yes
Is the year 1452 a year of peak sunspot activity? Yes

Thank you for using the YearChecker.
\end{Verbatim}
\vspace*{-.1in}
\caption{Sample ``{\tt YearCheckerMain}'' output featuring output from conditional logic checks.}
\label{mad}
\end{figure}

\section*{Checking the Correctness of Your Program and Writing}

As in the past assignments, you are provided with an automated tool for checking the quality of your source code. Please
note that the practical assignments do not require you to produce a writing document as you do in the laboratory
assignments. However, to check your Java source code you can started with the use of GatorGrader, type the command
\gatorgraderstart{} in your terminal window. Once this step completes you can type \gatorgradercheck{}. If your work
does not meet all of the assignment's requirements, then you will see the following output in your terminal:
\command{Overall, are there any mistakes in the assignment? Yes}. If you do have mistakes in your assignment, then you
will need to review GatorGrader's output, find the mistake, and try to fix it. Specifically, don't forget to add in the
required comments! If you are having trouble running GatorGrader locally, don't forget to ensure that you still transfer
all of your source code to GitHub. Please see the course instructor if you have questions about this step.

Once your program is building correctly, fulfilling at least some of the assignment's requirements, you should transfer
your files to GitHub using the \gitcommit{} and \gitpush{} commands. For example, if you want to signal that the
\mainprogramsource{} file has been changed and is ready for transfer to GitHub you would first type
\gitcommitmainprogram{} in your terminal, followed by typing \gitpush{} and checking to see that the transfer to GitHub
is successful. If you notice that transferring your code to GitHub did not work correctly, then please try to determine
why, asking a teaching assistant or the course instructor for help, if necessary.

After the course instructor enables \step{continuous integration} with a system called Travis CI, when you use the
\gitpush{} command to transfer your source code to your GitHub repository, Travis CI will initialize a \step{build} of
your assignment, checking to see if it meets all of the requirements. If both your source code and writing meet all of
the established requirements, then you will see a green \checkmark{} in the listing of commits in GitHub after awhile.
If your submission does not meet the requirements, a red \naughtmark{} will appear instead. The instructor will reduce a
student's grade for this assignment if the red \naughtmark{} appears on the last commit in GitHub immediately before the
assignment's due date. Yet, if the green \checkmark{} appears on the last commit in your GitHub repository, then you
satisfied all of the main checks. Unless you provide the course instructor with documentation of the extenuating
circumstances that you are facing, no late work will be considered towards your completion grade for this practical
assignment. You should aim to finish practical assignments on the day that they are assigned; please see the instructor
if you do not understand this policy.

Since this is another challenging practical assignment and you are still learning how to use the Java classes and
objects, don't become frustrated if you make a mistake. Instead, use your mistakes as an opportunity for learning both
about the necessary technology and the background and expertise of the other students in the class, the teaching
assistants, and the course instructor.

\noindent Students do not need to submit printed source code or technical writing for any assignment in this course.
Instead, this assignment invites you to submit, using GitHub, the following deliverables. Because this is a practical
assignment, you are not required to complete any technical writing.

\begin{enumerate}

\setlength{\itemsep}{0in}

\item A correct version of \mainprogramsource{} that meets all of the established source code requirements and produces
  the desired text-based output.

\item A correct version of \secondprogramsource{} that meets all of the established source code requirements and
  produces the desired text-based output.

\end{enumerate}

\section*{Evaluation of Your Practical Assignment}

Practical assignments are graded on a completion --- or ``checkmark'' --- basis.
If your GitHub repository has a \checkmark{} for the last commit before the
deadline then you will receive the highest possible grade for the assignment.
However, you will fail the assignment if you do not complete it correctly, as
evidenced by a red \naughtmark{} in your commit listing, by the set deadline for
completing the project. Please see the course instructor if you do not
understand how practical assignments are graded or you do not know how to
complete one of the specific tasks in this assignment.

\section*{Adhering to the Honor Code}

In adherence to the Honor Code, students should complete this practical assignment on an individual basis. While it is
appropriate for students in this class to have high-level conversations about the assignment, it is necessary to
distinguish carefully between the student who discusses the principles underlying a problem with others and the student
who produces assignments that are identical to, or merely variations on, someone else's work. Deliverables (e.g., the
Java source code) that are nearly identical to the work of others will be taken as evidence of violating the \mbox{Honor
Code}. Please see the course instructor if you have questions about this policy.

\end{document}
